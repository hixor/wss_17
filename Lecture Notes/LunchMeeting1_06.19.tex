\documentclass[11pt]{article}
\usepackage[utf8]{inputenc}	

\usepackage{amsmath,amsthm,amssymb,amscd}
\usepackage[margin=3cm]{geometry}
\usepackage{amsmath}
\newlength{\tabcont}
\setlength{\parindent}{0.0in}
\setlength{\parskip}{0.05in}
\parindent 0in
\parskip 12pt
\geometry{margin=1in, headsep=0.25in}
\theoremstyle{definition}
\newtheorem{defn}{Definition}
\newtheorem{reg}{Rule}
\newtheorem{exer}{Exercise}
\newtheorem{note}{Note}

\begin{document}

\title{Introduction to WSS '17 - Lecture 1, 9am-10am}

\thispagestyle{empty}
%%%%%%%%%%%%%%%%%%%%%%%%%%%%%%%%%%%%%%
%   TITLE
%%%%%%%%%%%%%%%%%%%%%%%%%%%%%%%%%%%%%%
\begin{center}
{\LARGE \bf Select Topics: 2pm}\\
\medskip
%%%%%%%%%%%%%%%%%%%%%%%%%%%%%%%%%%%%%%
%   DATE
%%%%%%%%%%%%%%%%%%%%%%%%%%%%%%%%%%%%%%
{\Large \today}\\
\smallskip
%%%%%%%%%%%%%%%%%%%%%%%%%%%%%%%%%%%%%%
%   COURSE
%%%%%%%%%%%%%%%%%%%%%%%%%%%%%%%%%%%%%%
{\large Wolfram Summer School 2017}
\end{center}

%%%%%%%%%%%%%%%%%%%%%%%%%%%%%%%%%%%%%%
%   ANNOUNCEMENTS
%%%%%%%%%%%%%%%%%%%%%%%%%%%%%%%%%%%%%%
\section*{Announcements}
\begin{itemize}
\item Announcement
\end{itemize}

\noindent\hrulefill


%%%%%%%%%%%%%%%%%%%%%%%%%%%%%%%%%%%%%%
%   AGENDA
%%%%%%%%%%%%%%%%%%%%%%%%%%%%%%%%%%%%%%

\section*{Agenda}
\begin{enumerate}
\item Agenda item
\end{enumerate}

\noindent\hrulefill

%%%%%%%%%%%%%%%%%%%%%%%%%%%%%%%%%%%%%%
%   RECAP
%%%%%%%%%%%%%%%%%%%%%%%%%%%%%%%%%%%%%%


%%%%%%%%%%%%%%%%%%%%%%%%%%%%%%%%%%%%%%
%   TODAY
%%%%%%%%%%%%%%%%%%%%%%%%%%%%%%%%%%%%%%

\section{What should you get out of this School?}

\subsection{Thinking Computationally}

\subsubsection{What does is mean to think about things computationally?}
\begin{itemize}
\item There’s raw computation, but also human ideas of computation
\item What is computation capable of?
\begin{itemize}
\item Simple rules, but complex results
\item But what can we as humans make of this?
\begin{itemize}
\item What can we achieve?
\item What is meaningful?
\end{itemize}
\item What is computation capable of?
\item Simple rules, but complex results
\item But what can we as humans make of this?
\begin{itemize}
\item What can we achieve?
\item What is meaningful?
\end{itemize}
\end{itemize}
\item Role of computer language design in serving as the interface between computation itself and human practice and application
\begin{itemize}
\item Thinking in terms of the language, as a way to organize one’s thoughts, is powerful
\begin{itemize}
\item This sort of discipline is a kind of focusing method, and is useful
\item This is a kind of experience and intuition
\end{itemize}
\end{itemize}
\end{itemize}

\subsubsection{For those who haven’t done original projects before This is a good place to start!}
\begin{itemize}
\item Usually, school projects already have a known ending. This is boring!
\end{itemize}

\subsubsection{How do you ACTUALLY PROUCE OUTPUT?}
\begin{itemize}
\item Doing something in order to write a paper is different from creating a shippable product
\end{itemize}

\subsection{Something to achieve:}
\begin{enumerate}
\item Get to have an awesome profile page on the community wiki!
\item Main thing is the project.
\item Homework project
\end{enumerate}



\subsection{Homework project: “Topic exploration”}
This should be a kind of exercise in communication.

\subsubsection{Guidelines}

Want it to be intuitive, interesting. 

\section{Resources}

\subsection{Data}

The Wolfram Data Repository has plenty of data that is ``lightly curated" - that is, not heavily processed, but okay.

\textbf{Interesting thing to do: Curate-a-thon - } Take a bunch of data and curate them in nice ways!

\subsection{Challenges}

Challenges can be found in the programming lab.

SW wants to hand out prizes for establishing facts about rule 30.

Ideas:
\begin{enumerate}
\item Do a simple classification - which rules is it most like?
\item Sequential parsing? That is, instead of just of the whole image?
\end{enumerate}

Challenge 2: Try to understand and explain Proof of Wolfram's Axiom for Boolean Algebra.

Ideas:
\begin{enumerate}
\item Turn it into a network or graph, with different colors for different variables
\item Think about the meaning of the equals sign
\end{enumerate}

\section{Neural Nets}

What kinds of primitives can we create symbolic forms for in order to build up neural nets?


\textbf{Science likes to have a narrative for humans to follow, so that it's meaningful. Does there exist a kind of narrative fallacy in science?}

\section{Universal Symbolic Languages}

Is there a way to have a language that could understand every human utterance? 

E.g., consider the sentence ``I want a piece of chocolate". Right now, WolframAlpha can understand `piece of chocolate', but not `I want'. 

How do we establish some kind of symbolic representation of human desires/emotions/ideas?

This is all just a matter of specifying exact meanings of words and symbols. English is too fuzzy. 

SW thinks that this has an interesting use case in AI communication - that is, how do we communicate a kind of ethical statement that is precise? This seems to be hopping around similar things as Constructer Theory (Evan)? That is, some sort of codification of morality...?


\section{Notes}

SW thinks that deep learning and cellular automata are fundamentally linked. There's a problem, though, since the CAs are finite and discrete in structure. Okay, well, just limit your weights and gradients to integers. Perhaps if you add enough layers, you can get something complex. This would be very nice.

Then, gradient descent would be finding neighbors that are discrete, rather than continuous.



\end{document}